\documentclass[11pt]{article}
\usepackage{amsmath,amssymb,enumerate,fancyhdr,calc}

\newcommand{\mmm}{1in}
\newcommand{\hhh}{28pt}

\usepackage[margin=\mmm]{geometry}

\addtolength{\topmargin}{\hhh}
\setlength{\textheight}{10.5in-\mmm-\mmm-\hhh}
\setlength{\parindent}{0in}

\pagestyle{fancy}

\newcommand{\dhl}{Dispersive Hydrodynamics Lab}
\newcommand{\ucb}{University of Colorado Boulder}
\newcommand{\appm}{Department of Applied Mathematics}
\newcommand{\me}{Grant Baker}
\newcommand{\ymd}{2017-05-04}

\title{\textbf{\dhl}\\ Water Table Experiment:\\ Data Collection and Image Processing}
\author{\textbf{\me}\\ \textbf{\appm}\\ \textbf{\ucb}}
\date{\textbf{\ymd}}

\begin{document}

\lhead{{\textbf{\me} \\ \textbf{\ymd}}}
\rhead{{\ucb \\ \appm \\ \dhl}}
\renewcommand{\headrulewidth}{1pt}
\renewcommand{\headheight}{39pt}

\maketitle

\section*{Overview}

In this document I will document the full process for data collection and image processing currently used by the water table team, consisting of Pat Sprenger, Adam Binswanger, and myself.\\

This process has three main stages: the experimental setup; the data collection; and the image processing. The experimental setup involves ensuring the water table and imaging equipment are set up properly. Data collection involves the process of using the camera and a laptop to collect images and store them properly. Image processing involves combining data from multiple images, removing noise and glare, and processing using Fourier Transform Profilometry to extract water height data.\\

\section*{The Experimental Setup}

\subsection*{The Water Table}

The setup involves a water table with a sluice gate, a water reservoir, and a pump. Ensure that the table is set up properly with the following steps:
\begin{enumerate}[(i)]
    \item Make sure the tubes are connected to the pump and the PVC apparatus, and the PVC pipes are covered in cloths and sponges are in the center of the rectangle.

    \item The sluice gate should be at the proper height (both sides), with both edges sealed watertight. The sealing is done with a watertight sealant, put on the sides of the gate and cut with a razor to the correct thickness.

    \item The hexagonal honeycomb flow straightener should be pushed to the bottom of the tank, and up against the gate.

    \item Ensure there is enough water in the tank (it should be nearly full) and there is enough titanium dioxide in the water to provide a sufficiently colored flow on the table. Typically, a few scoops is sufficient, and adding more typically does not improve coloration.
\end{enumerate}

\subsection*{Imaging Equipment}

The imaging equipment consists of the projector mounted above the table, a camera mounted near the projector, and a laptop to control the camera and store images.

\begin{enumerate}[(i)]
    \item The projector should be fitted to project on just the table, and focused to the undeformed water surface.

    \item The camera typically used is the Canon 7D Mark II. Mount it near the projector and power it with either a battery or the A/C power adapter. Make sure it has the zoom lens and the polarizing lens attached.

    \item The laptop used to run the projector and camera should be Windows or Mac, and should have the Canon EOS Utility software installed.

    \item Project a fringe pattern with the projector from the laptop. One can be generated with the {\tt generateFringePattern.m} script.

    \item Plug the camera in via USB and monitor it from the laptop. Focus the camera by looking at the live image on the laptop through the Canon software. If needed, further focus the projector as well. Rotate the polarizing lens to minimize glare on the water's surface, if needed.
\end{enumerate}

\section*{Data Collection}

Collecting data involves taking pictures with the camera and storing them on the laptop and on Google Drive.

\subsection*{Running the Experiment}

If the table and imaging equipment are set up properly, the process of running the experiment is straightforward.

\begin{enumerate}[(i)]
    \item Turn the lights off and close the blinds. Minimize all light coming from sources other than the projector.

    \item Take all images with the Canon EOS Utility. It is easiest if it is set to save the images directly on the laptop, and best if they are in the {\tt .RAW} file format.

    \item Take many images for each experiment, to ensure optimal performance from the averaging process.

    \item Take many reference images of the undeformed water surface.
\end{enumerate}

\subsection*{Data Storage}

So as to not lose data, it is important to store the images in two redundant locations.

\begin{enumerate}[(i)]
    \item Save all images collected in a directory titled the date.

    \item Create a text file in the same directory. Following a similar format to the text files that already exist, label the images in the text file.

    \item During that process, discard any unnecessary images.

    \item Upload the entire directory to the Google Drive folder.

    \item Copy the directory to the water table hard drive.
\end{enumerate}

\section*{Image Processing}

To extract height data from our images, we use Fourier Transform Profilometry, as outlined in {\it Cobelli 2009}.

\subsection*{Preprocessing}

\begin{enumerate}[(i)]
    \item First, use a raw image processing program to rotate or crop the images. Make sure each image in the data image set as well as the reference image set are processed in the same way. The current program used to do so is UFRaw. (Alternatively, write a script to batch process all images in this manner.)

    \item Use the {\tt imageaveraging.m} script to average all the images in each data set. This script both removes most of the glare from the image and employs a chi-square test to ensure the data is normal and that this averaging process is justified.

    \item Save the results in the {\tt images} directory as {\tt X\_data.png} and {\tt X\_ref.png}. Then run {\tt ./im.sh X} to set up the MATLAB script.
\end{enumerate}

\subsection*{Fourier Transform Profilometry}

We employ Fourier Transform Profilometry to extract height data, and use a number of filtering techniques to do so.

\begin{enumerate}[(i)]
    \item Adjust the parameters in the {\tt ftp\_X.m} script, where {\tt X} is the latest version of the FTP code. (The current version is {\tt v3}.)

    \item Run the script. It will produce height data. Adjust filtering parameters as necessary.

    \item If desired, save the height data.
\end{enumerate}



\end{document}
