\documentclass[12 pt]{article}
\usepackage{amsmath,amssymb}
\usepackage[margin = 1in]{geometry}

\newcommand{\theDate}{January 19, 2017} %put the date here
\newcommand{\yourName}{Pat Sprenger} % put your name here

\begin{document}

\begin{center}
Dispersive Hydrodynamics Lab Weekly Report \\
\yourName\\
\theDate
\end{center}

\section*{Goals} %In this section, you should summarize what you planned to accomplish this week. This can be brief, but should provide enough information so that other lab members can understand what you are working on currently. If you have many tasks that you planned to finish this week, a bulleted list may be helpful using the "enumerate" environment.

My goal this week is to demonstrate, contrary to what George Orwell taught me in \textit{1984}, that $2+2 = 4.$

\section*{Accomplishments}
% In this section, address which items from the previous section were accomplished this week. Here you can provide any figures or main results generated from your recent work. Any lengthy calculations should be included in a separate section.

Using my double plus mediocre addition skills, I accomplished my goal of showing $2+2 = 4$.

\section*{Questions} %Previous sections can be reworked to fit your personal liking, but this section is crucial. When starting research, unexpected questions arise and can slow progress and learning. This often is frustrating and makes research less enjoyable. Use this section to organize your questions regarding your work--sometimes just writing out your thoughts can help you reorganize and look at problems differently.

Why does my result contradict Winston and Big Brother's? They claim that $2+2 = 5$.

\section*{Calculations/code}
%Use this section to store any calculations/code. I know using tex to write up your calculations is time consuming, but it will make your life easier when you are writing a paper. Including any code here will make it easier to find should you misremember where it is saved (this happens to me all too often).
Using the distributive property of multiplication on the real numbers, we have
\begin{align*}
2+2 = 2(1+1)
\end{align*}
Since addition is associative
\begin{align*}
2(1+1) = 2*2 = 4.
\end{align*}
\end{document}
